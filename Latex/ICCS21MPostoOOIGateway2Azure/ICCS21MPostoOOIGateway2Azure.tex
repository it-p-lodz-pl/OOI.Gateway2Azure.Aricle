\documentclass[runningheads]{llncs}
\usepackage{graphicx}

\begin{document}

\title{Object-Oriented Internet\\ Cloud Interoperability}
\titlerunning{OOI - Cloud Interoperability}

\author{Mariusz Postół\inst{1}\orcidID{0000-0002-9669-0565}  \and Piotr Szymczak\inst{1}\orcidID{1111-2222-3333-4444} }
\authorrunning{M. Postół et al.}

\institute
{ Institute of Information Technology, Lodz University of Technology, Łódź, Poland \\
    \email{mailto:mariusz.postol@p.lodz.pl}\\
    \url{http://www.it.p.lodz.pl}
}

\maketitle              % typeset the header of the contribution


\begin{abstract}

    Optimization of the industrial processes requires further research on the integration of machine-centric systems with human-centric cloud-based services in the context of new emerging disciplines, namely Industry 4.0 and Industrial Internet of Things. This research aims at working out a new generic architecture and deployment scenario applicable to this integration. The reactive interoperability relationship of the interconnected nodes is proposed to deal with the network traffic propagation asymmetry or assets' mobility. Described solution based on the OPC Unified Architecture international standard relaxes issues related to the real-time multi-vendor environment. The discussion addressing the generic architecture concludes that the embedded gateway software part best suits all requirements. To promote separation of concerns and reusability, the proposed architecture of the embedded gateway has been implemented as a composable part of a selected OPC UA PubSub framework.

    The proposals are backed by proof of concept reference implementations confirming the possibility to integrate selected cloud services with the cyber-physical system interconnected as one whole atop of the OPC UA by applying the proposed architecture and deployment scenario. It is contrary to interconnecting cloud services with a selected OPC UA server limiting the PubSub role to data export only.

    \keywords{Industry 4.0 \and Internet of Things \and Object-Oriented Internet \and Cloud Computing \and Industrial communication \and Reactive Networking (RxNetworking)\and Machine to Machine Communication \and OPC Unified Architecture \and Azure}

\end{abstract}

\section{First Section}
\subsection{A Subsection Sample}

Please note that the first paragraph of a section or subsection is not indented. The first paragraph that follows a table, figure, equation etc. does not need an indent, either.

Subsequent paragraphs, however, are indented.

\subsubsection{Sample Heading (Third Level)}

Only two levels of headings should be numbered. Lower level headings remain unnumbered; they are formatted as run-in headings.

\paragraph{Sample Heading (Fourth Level)}

The contribution should contain no more than four levels of headings.

\bibliography{ICCS21MPostoOOIGateway2Azure}
\bibliographystyle{splncs04}

\end{document}
