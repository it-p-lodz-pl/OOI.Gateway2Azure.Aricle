% Options for packages loaded elsewhere
\PassOptionsToPackage{unicode}{hyperref}
\PassOptionsToPackage{hyphens}{url}
%
\documentclass[
]{article}
\usepackage{lmodern}
\usepackage{amssymb,amsmath}
\usepackage{ifxetex,ifluatex}
\ifnum 0\ifxetex 1\fi\ifluatex 1\fi=0 % if pdftex
  \usepackage[T1]{fontenc}
  \usepackage[utf8]{inputenc}
  \usepackage{textcomp} % provide euro and other symbols
\else % if luatex or xetex
  \usepackage{unicode-math}
  \defaultfontfeatures{Scale=MatchLowercase}
  \defaultfontfeatures[\rmfamily]{Ligatures=TeX,Scale=1}
\fi
% Use upquote if available, for straight quotes in verbatim environments
\IfFileExists{upquote.sty}{\usepackage{upquote}}{}
\IfFileExists{microtype.sty}{% use microtype if available
  \usepackage[]{microtype}
  \UseMicrotypeSet[protrusion]{basicmath} % disable protrusion for tt fonts
}{}
\makeatletter
\@ifundefined{KOMAClassName}{% if non-KOMA class
  \IfFileExists{parskip.sty}{%
    \usepackage{parskip}
  }{% else
    \setlength{\parindent}{0pt}
    \setlength{\parskip}{6pt plus 2pt minus 1pt}}
}{% if KOMA class
  \KOMAoptions{parskip=half}}
\makeatother
\usepackage{xcolor}
\IfFileExists{xurl.sty}{\usepackage{xurl}}{} % add URL line breaks if available
\IfFileExists{bookmark.sty}{\usepackage{bookmark}}{\usepackage{hyperref}}
\hypersetup{
  hidelinks,
  pdfcreator={LaTeX via pandoc}}
\urlstyle{same} % disable monospaced font for URLs
\setlength{\emergencystretch}{3em} % prevent overfull lines
\providecommand{\tightlist}{%
  \setlength{\itemsep}{0pt}\setlength{\parskip}{0pt}}
\setcounter{secnumdepth}{-\maxdimen} % remove section numbering

\date{}

\begin{document}

\hypertarget{azure-main-technology-features}{%
\section{Azure Main Technology
Features}\label{azure-main-technology-features}}

\hypertarget{services}{%
\subsection{Services}\label{services}}

Deployment of the hybrid solution providing interoperability of the
\textbf{machine-centric} cyber-physical systems designed atop of M2M
reactive communication and emerging cloud computing as a
\textbf{human-centric} front-end requires decisions addressing the
selection of the services supporting web user interface capable to
expose real-time process data. In this context, the service is any
autonomous (with own identity) software component or module that is
interfacing with selected cyber-physical systems for data collection,
analysis, and also remote control. Microsoft Azure is a cloud-based
product. It offers a vast variety of services. This virtual environment
handles an unlimited number of users and devices organized using a
solution concept. The solution aggregates users, devices, services, and
required additional resources scoping on a selected scenario. The
solution serves as a context that provides a scope to the identifiers
(the names of devices, users, process data entities, etc) inside it.
Solutions are used to organize deployment entities into logical groups
and prevent identity collisions.

The \texttt{IoT\ Central} service provides a process data visualization
user interface. To make this interface meaningful metadata called device
template is used to describe devices.

Following the assumption that interconnection between the cyber-physical
system and cloud services is designed based on the gateway concept, a
middleware must be considered as a coupler. It must be interconnected
with the cyber-physical system using an in-band protocol adhering to
communications requirements (i.e.~protocol profile, data encoding, time
relationships, etc.) governing communication of the parts making it up.
At the same time, it must support back-and-forth data transfer to the
cloud using out-of-band native for the cloud services. The transfer
process requires data conversion from source to destination encoding.
The \texttt{IoT\ Hub} is a service hosted in the cloud that supports
\texttt{IoT\ Central} providing a robust messaging solution - it acts as
a central message hub for bi-directional communication. This
communication is transparent, i.e.~it is not data types aware allowing
any devices to exchange any kind of data. This service is responsible to
manage the devices' identity and it offers the following protocol
stacks: AMQP, MQTT, HTTPS.

Before process data can be exposed using a web user interface the data
source must be associated with an appropriate solution and validated to
make sure that the security rules are not violated. It is hard to assume
that the security rules governing the cyber-physical system may also
apply to the cloud-based services. In the gateway scenario, they can be
mapped on each other or entirely independent. The
\texttt{IoT\ Hub\ Device\ Provisioning\ Service} (DPS) is a helper
service for \texttt{IoT\ Hub} that enables devices' connection process
management, upon device providing valid identity attestation it assigns
the device to an appropriate \texttt{IoT\ Hub} instance and returns to
the device connection parameters, which allow direct connection with
given \texttt{IoT\ Hub} service. The device proceeds to use the same
attestation in \texttt{IoT\ Hub} connection and based on it, is granted
authorization to selected resources and operations including but not
limited to data transfer updating the user interface.

It is worth stressing that interaction of the offered by the Azure
services can be configured flexibly, and as a result, the presented
above selection of services must be recognized as an example only. The
\texttt{IoT\ Central} can be also seamlessly integrated with other
services as needed. The following services could also be considered to
build cloud-based automation solution:

\begin{itemize}
\tightlist
\item
  \texttt{Industrial\ IoT} - discovering OPC UA enabled servers in a
  factory network and register them in Azure IoT Hub implemented using
  \texttt{IoT\ Edge}
\item
  \texttt{Digital\ Twins} - managing the graph of digital twins, which
  are to represent some real-world process or entity
\end{itemize}

\texttt{Industrial\ IoT} promotes OPC UA client/server archetype used to
achieve direct and interactive interoperability implemented using
\texttt{IoT\ Edge} services that allow extracting initial data
processing to local premises based on the edge concept.
\texttt{Digital\ Twins} is an emerging concept to use an observer to
replicate selected process state and behavior. The possibility to add
value as a result of using these services must be subject to further
research.

\hypertarget{data-interchange}{%
\subsection{Data Interchange}\label{data-interchange}}

System components interoperability means the necessity of the
information exchange between them. The main challenge of
interoperability implementation is that information is abstract -- it is
knowledge describing the process in concern state and behavior,
e.g.~temperature in a boiler, a car speed, an account balance, etc.
Obviously, abstraction cannot be processed by the cyber-physical
machines. It is also impossible to transfer abstraction from one place
to another over the network.

Fortunately, computer science offers a workaround to address that
impossibility - the information must be represented as a binary stream.
In consequence, we can usually use both ones as interchangeable terms
while talking about ICT systems. Unfortunately, these terms must be
precisely observed in the context of further discussion, because we must
be aware of the fact that the same information could have many different
but equivalent representations. In other words, the same information can
be represented by a vast variety of different binary patterns. For
example, numbers may be represented using 2's Complement and
Floating-Point binary representations.

It should be nothing new for us, as it is obvious that the same
information printed as a text in regional newspapers in English, German,
Polish, etc. does not resemble one another, but the text meaning should
always be the same. To understand a newspaper we must learn the
appropriate language. To understand the binary data we must have defined
a data type -- a description of how to create an appropriate bits
pattern (syntax) and rules needed to assign the information (semantics),
i.e.~make any correct bitstream meaningful. Concluding, to make two
systems interoperable, a semantic-context must be established. The type
plays the role of metadata, a set of data that describes other data.
Metadata term is frequently used if the semantic-context is defined
using a native language to select built-in types engaging a
general-purpose graphical user interface.

Using the data type definitions to describe information interchanged
between communicating parties allows:

\begin{itemize}
\tightlist
\item
  Development against a type definition of the user interface
\item
  Implementation of the functionality of the bitstreams conversion in
  advance
\end{itemize}

Having defined types in advance, a gateway may provide dedicated
conversions functionality, i.e.~replacing bitstream used by the
cyber-physical system by equivalent one for the cloud-based services.
The Azure offers a vast variety of built-in types ready to be used in
common cases, but not necessarily there are equivalent counterparts in
use by the cyber-physical system. Additionally, the data conversion must
address the following issues:

\begin{itemize}
\tightlist
\item
  usually to covert data from source to destination representation, the
  middleware software native types must be used
\item
  if the out of the box set of types is not capable of fulfilling more
  demanding needs, custom data types must be defined
\end{itemize}

Although the data conversion is a run-time gateway task the
implementation of the conversion algorithms must be recognized as an
engineering task, and therefore this topic is not considered for further
discussion.

In \texttt{IoT\ Central} a cyber-physical system is represented as a set
of devices. The characteristics and behaviors of each device kind are
described by the device template. This Device Template (DT) contains
also metadata describing the data (called telemetry) exchanged over the
wire with the cyber-physical system called Device Capability Model
(DCM). Additionally, the DT contains properties, customization, and
views definitions used by the service locally. As an option, DCM
expressed as a JSON-LD can be imported into a Device Template.
\texttt{IoT\ Central} allows also to create and edit a DCM using the
dedicated web UI. A JSON file containing DCM can be derived from an
information model used as a foundation to establish the semantic-context
applied to achieve interoperability of the devices interconnected as the
cyber-physical system. DCM development against any external information
model is a design-time task and should be supported by dedicated
development tools. In any case, the data interchanged between the cloud
and the gateway must be compliant with the DCM, hence the middleware
software must be aware of conversions that must be applied to achieve
this interoperability.

\hypertarget{connectivity}{%
\subsection{Connectivity}\label{connectivity}}

From the cloud side, it is proposed to employ the \texttt{IoT\ Hub}
service to handle the network traffic targeting the cyber-physical
system. This service offers profiles of the AMQP, MQTT, HTTPS protocol
stacks. In any case, process data (telemetry) is transparently
transferred back-and-forth to the upper layer \texttt{IoT\ Central}
service. Hence, the payload formatting is determined by the DCM
associated with the \texttt{IoT\ Central} solution. All the mentioned
protocols are standard ones. Consequently, it is possible to apply any
available implementation compliant with an appropriate specification to
achieve connectivity. In this case, all parameters required to establish
connectivity and security-context is up to the external software
responsibility. Alternatively, the API offered by the dedicated
libraries may be used. Using this API the configuration process can be
reduced significantly. Using these libraries, the selection of the
communication protocol has an indirect impact on the interoperability
features, including performance. The connectivity with
\texttt{IoT\ Hub}, for example, can be obtained using:

\begin{itemize}
\tightlist
\item
  \texttt{Microsoft.Azure.Devices} - Service SDK for Azure IoT Devices
\item
  \texttt{Microsoft.Azure.Devices.Client}- Device SDK for Azure IoT Hub
\item
  \texttt{Microsoft.Azure.Devices.Shared} - Common code for Azure IoT
  Device and Service SDKs
\item
  \texttt{Microsoft.Azure.Devices.Provisioning.Client} - Provisioning
  Device Client SDK for Azure IoT Devices
\item
  \texttt{Microsoft.Azure.Devices.Provisioning.Transport.Amqp} -
  Provisioning Device Client AMQP Transport for Azure IoT Devices
\item
  \texttt{Microsoft.Azure.Devices.Provisioning.Transport.Http} -
  Provisioning Device Client Http Transport for Azure IoT Devices
\item
  \texttt{Microsoft.Azure.Devices.Provisioning.Transport.Mqtt} -
  Provisioning Device Client MQTT Transport for Azure IoT Devices
\end{itemize}

\end{document}
