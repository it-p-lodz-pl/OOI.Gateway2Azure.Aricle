% Options for packages loaded elsewhere
\PassOptionsToPackage{unicode}{hyperref}
\PassOptionsToPackage{hyphens}{url}
%
\documentclass[
]{article}
\usepackage{lmodern}
\usepackage{amssymb,amsmath}
\usepackage{ifxetex,ifluatex}
\ifnum 0\ifxetex 1\fi\ifluatex 1\fi=0 % if pdftex
  \usepackage[T1]{fontenc}
  \usepackage[utf8]{inputenc}
  \usepackage{textcomp} % provide euro and other symbols
\else % if luatex or xetex
  \usepackage{unicode-math}
  \defaultfontfeatures{Scale=MatchLowercase}
  \defaultfontfeatures[\rmfamily]{Ligatures=TeX,Scale=1}
\fi
% Use upquote if available, for straight quotes in verbatim environments
\IfFileExists{upquote.sty}{\usepackage{upquote}}{}
\IfFileExists{microtype.sty}{% use microtype if available
  \usepackage[]{microtype}
  \UseMicrotypeSet[protrusion]{basicmath} % disable protrusion for tt fonts
}{}
\makeatletter
\@ifundefined{KOMAClassName}{% if non-KOMA class
  \IfFileExists{parskip.sty}{%
    \usepackage{parskip}
  }{% else
    \setlength{\parindent}{0pt}
    \setlength{\parskip}{6pt plus 2pt minus 1pt}}
}{% if KOMA class
  \KOMAoptions{parskip=half}}
\makeatother
\usepackage{xcolor}
\IfFileExists{xurl.sty}{\usepackage{xurl}}{} % add URL line breaks if available
\IfFileExists{bookmark.sty}{\usepackage{bookmark}}{\usepackage{hyperref}}
\hypersetup{
  hidelinks,
  pdfcreator={LaTeX via pandoc}}
\urlstyle{same} % disable monospaced font for URLs
\setlength{\emergencystretch}{3em} % prevent overfull lines
\providecommand{\tightlist}{%
  \setlength{\itemsep}{0pt}\setlength{\parskip}{0pt}}
\setcounter{secnumdepth}{-\maxdimen} % remove section numbering

\date{}

\begin{document}

\hypertarget{introduction}{%
\section{Introduction}\label{introduction}}

All the time, the Information and Communication Technology is providing
society with a vast variety of new distributed applications aimed at
micro and macro optimization of the industrial processes. Obviously, the
design foundation of this kind of application has to focus primarily on
communication technologies. Based on the role humans take while using
these applications they can be grouped as follows:

\begin{itemize}
\tightlist
\item
  \textbf{human-centric} - information origin or ultimate information
  destination is an operator,
\item
  \textbf{machine-centric} - information creation, consumption,
  networking, and processing are achieved entirely without human
  interaction.
\end{itemize}

A typical \textbf{human-centric} approach is a web-service supporting,
for example, a web user interface (UI) to monitor conditions, and manage
millions of devices and their data in a typical cloud-based IoT
approach. It is characteristic that, in this case, any uncertainty and
necessity to make a decision can be relaxed by human interaction.
Coordination of robots behavior in a work-cell (automation islands) is a
\textbf{machine-centric} example. In this case, it is essential that any
human interaction is impractical or even impossible. This
interconnection scenario requires the machine to machine communication
(M2M) demanding multi-vendor devices integration.

From the M2M communication concept, a broader concept of a smart factory
can be derived. In this concept, the mentioned robots are only executive
assets of an integrated supervisory control system responsible for macro
optimization of an industrial process composed into one whole.
Deployment of the smart factory concept requires a hybrid solution and
interoperability of the mentioned above heterogeneous environments. This
approach is called the fourth industrial revolution and coined as
Industry 4.0. It is worth stressing that machines - or more general
assets - interconnection is not enough, and additionally, assets
interoperability has to be expected for the deployment of this concept.
In this case, muli-vendor integration makes communication
standardization especially important, namely it is required that the
payload of the message is standardized to be factored on the gathering
site and consumed on the ultimate destination site.

Highly-distributed solutions used to control real-time process
aggregating islands of automation (e.g.~virtual power plants producing
renewable energy) additionally must leverage public communication
infrastructure, namely the Internet. Internet is a demanding environment
for highly distributed process control applications designed atop the
M2M communication paradigm because

\begin{itemize}
\tightlist
\item
  it is a globally shareable environment and can be also used by
  malicious users
\item
  it offers only non-deterministic communication making integration of
  islands of automation designed against the real-time requirements a
  difficult task
\end{itemize}

Today both obstacles can be overcome, and as examples, we have bank
account remote control and voice over IP in daily use. The first
application must be fine-tuned in the context of data security, and the
second is very sensitive on time relationships. Similar approaches could
be applied to adopt the well known in process control industry concepts:

\begin{itemize}
\tightlist
\item
  Human Machine Interface (HNI)
\item
  Supervisory Control and Data Acquisition (SCADA)
\item
  Distributed Control Systems (DCS)
\end{itemize}

A detailed examination of these solutions is far beyond the scope of
this article. It is only worth stressing that, by design, all of them
are designed on the foundation of interactive communication. Interactive
communication is based on a data polling foundation. In this case, the
application must follow the interactive behavioral model, because it
actively polls the data source for more information by pulling data from
a sequence that represents the process state in time. The application is
active in the data retrieval process - it controls the pace of the
retrieval by sending the requests at its convenience. Such a polling
pattern is similar to visiting the books shop and checking out a book.
After you are done with the book, you pay another visit to check out
another one. If the book is not available you must wait, but you may
read what you selected. The client/server archetype is well suited for
the mentioned above applications.

After dynamically attaching a new island of automation the control
application (responsible for the data pulling) must be reconfigured for
this interoperability scenario. In other words, in this case, the
interactive relationship cannot be directly applied because the control
application must be informed on how to pull data from a new source. As a
result, a plug and produce scenario cannot be seamlessly applied. A
similar drawback must be overcome if for security reasons suitable
protection methods have been applied to make network traffic propagation
asymmetric. It is accomplished using intermediary devices, for example,
firewalls, to enforce traffic selective availability based on
predetermined security rules against unauthorized access.

Going further, we shall assume that islands of automation are mobile,
e.g.~autonomous cars passing a supervisory controlled service area. In
this case, the behavior of the interconnected assets is particularly
important concerning the environment in which they must interact. This
way we have entered the Internet of Things domain of Internet-based
applications.

If we must bother with the network traffic propagation asymmetry or
mobility of the asset network attachment-points the reactive
relationship could relax the problems encountered while the interactive
approach is applied. In this case, the sessionless publisher-subscriber
communication archetype is a typical pattern to implement the abstract
reactive interoperability paradigm. The sessionless archetype is a
message distribution scenario where senders of messages, called
publishers, do not send them directly to specific receivers, called
subscribers, but instead, categorize the published messages into topics
without knowledge about which subscribers if any, there may be.
Similarly, subscribers express interest in one or more topics and only
receive messages that are of interest, without knowledge about which
publishers, if any, there are. In this scenario, the publishers and
subscribers are loosely coupled.i.e they are decoupled in time, space
and synchronization \cite{RefWorks:doc:5c44e246e4b0591b15ea9e59}.

If the \textbf{machine-centric} interoperability - making up islands of
automation - must be monitored and/or controlled by a supervisory system
cloud computing concept may be recognized as a beneficial solution to
replace or expand the mentioned above applications, i.e.~HMI, SCADA,
DCS, etc. Cloud computing is a method to provide a requested
functionality as a set of services. There are many examples that cloud
computing is useful to reduce costs and increase robustness. It is also
valuable in case the process data must be exposed to many stakeholders.
Following this idea and offering control systems as a service, there is
required a mechanism created on the service concept and supporting
abstraction and virtualization - two main pillars of the cloud computing
paradigm. In the cloud computing concept, virtualization is recognized
as the possibility to share the services by many users, and abstraction
hides implementation details.

Deployment of the hybrid solution providing interoperability of the
\textbf{machine-centric} cyber-physical systems and
\textbf{human-centric} cloud-based front-end can be implemented applying
the following scenarios:

\begin{itemize}
\tightlist
\item
  \textbf{direct interconnection} - cloud-based dedicated communication
  services allow to attache it to the cyber-physical system making up a
  consistent M2M communication network using an in-bound protocol stack
\item
  \textbf{gateway based interconnection} - typical build-in
  communication services allows to attache the cloud computing to the
  cyber-physical system using an out-of-bound protocol stack
\end{itemize}

By design, the direct approach requires that the cloud has to be
compliant with the interoperability standard the cyber-physical system
is built around - it becomes a consistent part of the cyber-physical
system. Data models, roles, and responsibility differences of both
solutions make this approach impractical and imposable to be applied in
typical cases. A more detailed description is covered by the section
\texttt{Azure\ to\ Sensors\ (A2S)\ connectivity\ deployment}.

This article addresses further research on the integration of the
cyber-physical systems in the context of new emerging disciplines,
i.e.~Industry 4.0 (I4.0) and the Internet of Things (IoT). The new
architecture is proposed for integration of the multi-vendor
\textbf{machine-centric} cyber-physical system designed atop of M2M
reactive communication and emerging cloud computing as a
\textbf{human-centric} front-end. To support the multi-vendor
environment OPC Unified Architecture interoperability standard has been
selected. The proposals are backed by proof of concept reference
implementations. Prototyping addresses Microsoft Azure Cloud as an
example. The prototyping outcome has been just published on GitHub as
the open-source (MIT licensed). The proposed solutions have been
harmonized with the more general concept called the Object-Oriented
Internet.

The main goal of this article is to provide proof that:

\begin{enumerate}
\def\labelenumi{\arabic{enumi}.}
\tightlist
\item
  reactive interoperability M2M communication based on the OPC UA
  standard can be implemented as a powerful standalone library without
  dependency on the Client/Server session-oriented archetype
\item
  Azure interoperability can be implemented as an external part
  employing out-of-band communication without dependency on the OPC UA
  implementation
\item
  the proposed generic architecture allows that the gateway
  functionality is composable at run-time - no programming required
\end{enumerate}

The remainder of this paper is structured as follows. Sect.
\texttt{Azure\ Main\ Technology\ \ Features} analyzes data presentation
user interface, available native communication services, and
data/metadata model offered by the Microsoft Azure. The discussion
covered by this section is the foundation for selecting services
utilized to expose process data and suitable protocol stack to support
interconnection. In Sect.
\texttt{Object-Oriented\ Internet\ Main\ Technology\ Features} the
discussion focuses on the generic architecture that is to be used as a
foundation for further decisions addressing the systematic design of the
interoperability of the cyber-physical systems and cloud-based
front-end. Sect.
\texttt{Azure\ to\ Sensors\ (A2S)\ connectivity\ deployment\ (field\ level\ connectivity)}
presents the proposed open and reusable software model. It promotes a
reactive interoperability pattern and a generic approach to establishing
interoperability-context. A reference implementation of this archetype
is described in Sect. \texttt{Gateway\ implementation}. The most
important findings and future work are summarized in Sect.
\texttt{Conclusions}.

\hypertarget{consider-to-add}{%
\subsection{Consider to add}\label{consider-to-add}}

\hypertarget{scope}{%
\subsubsection{Scope}\label{scope}}

What we must do to prove the goal have been achieved. Extent or range of
development, view, outlook, application, operation, effectiveness, etc.

\hypertarget{related-work}{%
\subsubsection{Related work}\label{related-work}}

Any information about available reusable deliverables related to this
work.

\end{document}
