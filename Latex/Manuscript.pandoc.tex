% Options for packages loaded elsewhere
\PassOptionsToPackage{unicode}{hyperref}
\PassOptionsToPackage{hyphens}{url}
%
\documentclass[
]{article}
\usepackage{lmodern}
\usepackage{amssymb,amsmath}
\usepackage{ifxetex,ifluatex}
\ifnum 0\ifxetex 1\fi\ifluatex 1\fi=0 % if pdftex
  \usepackage[T1]{fontenc}
  \usepackage[utf8]{inputenc}
  \usepackage{textcomp} % provide euro and other symbols
\else % if luatex or xetex
  \usepackage{unicode-math}
  \defaultfontfeatures{Scale=MatchLowercase}
  \defaultfontfeatures[\rmfamily]{Ligatures=TeX,Scale=1}
\fi
% Use upquote if available, for straight quotes in verbatim environments
\IfFileExists{upquote.sty}{\usepackage{upquote}}{}
\IfFileExists{microtype.sty}{% use microtype if available
  \usepackage[]{microtype}
  \UseMicrotypeSet[protrusion]{basicmath} % disable protrusion for tt fonts
}{}
\makeatletter
\@ifundefined{KOMAClassName}{% if non-KOMA class
  \IfFileExists{parskip.sty}{%
    \usepackage{parskip}
  }{% else
    \setlength{\parindent}{0pt}
    \setlength{\parskip}{6pt plus 2pt minus 1pt}}
}{% if KOMA class
  \KOMAoptions{parskip=half}}
\makeatother
\usepackage{xcolor}
\IfFileExists{xurl.sty}{\usepackage{xurl}}{} % add URL line breaks if available
\IfFileExists{bookmark.sty}{\usepackage{bookmark}}{\usepackage{hyperref}}
\hypersetup{
  hidelinks,
  pdfcreator={LaTeX via pandoc}}
\urlstyle{same} % disable monospaced font for URLs
\setlength{\emergencystretch}{3em} % prevent overfull lines
\providecommand{\tightlist}{%
  \setlength{\itemsep}{0pt}\setlength{\parskip}{0pt}}
\setcounter{secnumdepth}{-\maxdimen} % remove section numbering

\date{}

\begin{document}

\hypertarget{introduction}{%
\section{Introduction}\label{introduction}}

All the time, the Information and Communication Technology is providing
society with a vast variety of new distributed applications aimed at
micro and macro optimization of the industrial processes. Obviously, the
design foundation of this kind of application has to focus primarily on
communication technologies. Based on the role humans take while using
these applications they can be grouped as follows:

\begin{itemize}
\tightlist
\item
  \textbf{human-centric} - information origin or ultimate information
  destination is an operator,
\item
  \textbf{machine-centric} - information creation, consumption,
  networking, and processing are achieved entirely without human
  interaction.
\end{itemize}

A typical \textbf{human-centric} approach is a web-service supporting,
for example, a web user interface (UI) to monitor conditions, and manage
millions of devices and their data in a typical cloud-based IoT
approach. It is characteristic that, in this case, any uncertainty and
necessity to make a decision can be relaxed by human interaction.
Coordination of robots behavior in a work-cell (automation islands) is a
\textbf{machine-centric} example. In this case, it is essential that any
human interaction is impractical or even impossible. This
interconnection scenario requires the machine to machine communication
(M2M) demanding multi-vendor devices integration.

From the M2M communication concept, a broader concept of a smart factory
can be derived. In this concept, the mentioned robots are only executive
assets of an integrated supervisory control system responsible for macro
optimization of an industrial process composed into one whole.
Deployment of the smart factory concept requires a hybrid solution and
interoperability of the mentioned above heterogeneous environments. This
approach is called the fourth industrial revolution and coined as
Industry 4.0. It is worth stressing that machines - or more general
assets - interconnection is not enough, and additionally, assets
interoperability has to be expected for the deployment of this concept.
In this case, muli-vendor integration makes communication
standardization especially important, namely it is required that the
payload of the message is standardized to be factored on the gathering
site and consumed on the ultimate destination site.

Highly-distributed solutions used to control real-time process
aggregating islands of automation (e.g.~virtual power plants producing
renewable energy) additionally must leverage public communication
infrastructure, namely the Internet. Internet is a demanding environment
for highly distributed process control applications designed atop the
M2M communication paradigm because

\begin{itemize}
\tightlist
\item
  it is a globally shareable environment and can be also used by
  malicious users
\item
  it offers only non-deterministic communication making integration of
  islands of automation designed against the real-time requirements a
  difficult task
\end{itemize}

Today both obstacles can be overcome, and as examples, we have bank
account remote control and voice over IP in daily use. The first
application must be fine-tuned in the context of data security, and the
second is very sensitive on time relationships. Similar approaches could
be applied to adopt the well known in process control industry concepts:

\begin{itemize}
\tightlist
\item
  Human Machine Interface (HNI)
\item
  Supervisory Control and Data Acquisition (SCADA)
\item
  Distributed Control Systems (DCS)
\end{itemize}

A detailed examination of these solutions is far beyond the scope of
this article. It is only worth stressing that, by design, all of them
are designed on the foundation of interactive communication. Interactive
communication is based on a data polling foundation. In this case, the
application must follow the interactive behavioral model, because it
actively polls the data source for more information by pulling data from
a sequence that represents the process state in time. The application is
active in the data retrieval process - it controls the pace of the
retrieval by sending the requests at its convenience. Such a polling
pattern is similar to visiting the books shop and checking out a book.
After you are done with the book, you pay another visit to check out
another one. If the book is not available you must wait, but you may
read what you selected. The client/server archetype is well suited for
the mentioned above applications.

After dynamically attaching a new island of automation the control
application (responsible for the data pulling) must be reconfigured for
this interoperability scenario. In other words, in this case, the
interactive relationship cannot be directly applied because the control
application must be informed on how to pull data from a new source. As a
result, a plug and produce scenario cannot be seamlessly applied. A
similar drawback must be overcome if for security reasons suitable
protection methods have been applied to make network traffic propagation
asymmetric. It is accomplished using intermediary devices, for example,
firewalls, to enforce traffic selective availability based on
predetermined security rules against unauthorized access.

Going further, we shall assume that islands of automation are mobile,
e.g.~autonomous cars passing a supervisory controlled service area. In
this case, the behavior of the interconnected assets is particularly
important concerning the environment in which they must interact. This
way we have entered the Internet of Things domain of Internet-based
applications.

If we must bother with the network traffic propagation asymmetry or
mobility of the asset network attachment-points the reactive
relationship could relax the problems encountered while the interactive
approach is applied. In this case, the sessionless publisher-subscriber
communication archetype is a typical pattern to implement the abstract
reactive interoperability paradigm. The sessionless archetype is a
message distribution scenario where senders of messages, called
publishers, do not send them directly to specific receivers, called
subscribers, but instead, categorize the published messages into topics
without knowledge about which subscribers if any, there may be.
Similarly, subscribers express interest in one or more topics and only
receive messages that are of interest, without knowledge about which
publishers, if any, there are. In this scenario, the publishers and
subscribers are loosely coupled.i.e they are decoupled in time, space
and synchronization \cite{RefWorks:doc:5c44e246e4b0591b15ea9e59}.

If the \textbf{machine-centric} interoperability - making up islands of
automation - must be monitored and/or controlled by a supervisory system
cloud computing concept may be recognized as a beneficial solution to
replace or expand the mentioned above applications, i.e.~HMI, SCADA,
DCS, etc. Cloud computing is a method to provide a requested
functionality as a set of services. There are many examples that cloud
computing is useful to reduce costs and increase robustness. It is also
valuable in case the process data must be exposed to many stakeholders.
Following this idea and offering control systems as a service, there is
required a mechanism created on the service concept and supporting
abstraction and virtualization - two main pillars of the cloud computing
paradigm. In the cloud computing concept, virtualization is recognized
as the possibility to share the services by many users, and abstraction
hides implementation details.

Deployment of the hybrid solution providing interoperability of the
\textbf{machine-centric} cyber-physical systems and
\textbf{human-centric} cloud-based front-end can be implemented applying
the following scenarios:

\begin{itemize}
\tightlist
\item
  \textbf{direct interconnection} - cloud-based dedicated communication
  services allow to attache it to the cyber-physical system making up a
  consistent M2M communication network using an in-bound protocol stack
\item
  \textbf{gateway based interconnection} - typical build-in
  communication services allows to attache the cloud computing to the
  cyber-physical system using an out-of-bound protocol stack
\end{itemize}

By design, the direct approach requires that the cloud has to be
compliant with the interoperability standard the cyber-physical system
is built around - it becomes a consistent part of the cyber-physical
system. Data models, roles, and responsibility differences of both
solutions make this approach impractical and imposable to be applied in
typical cases. A more detailed description is covered by the section
\texttt{Azure\ to\ Sensors\ (A2S)\ connectivity\ deployment}.

This article addresses further research on the integration of the
cyber-physical systems in the context of new emerging disciplines,
i.e.~Industry 4.0 (I4.0) and the Internet of Things (IoT). The new
architecture is proposed for integration of the multi-vendor
\textbf{machine-centric} cyber-physical system designed atop of M2M
reactive communication and emerging cloud computing as a
\textbf{human-centric} front-end. To support the multi-vendor
environment OPC Unified Architecture interoperability standard has been
selected. The proposals are backed by proof of concept reference
implementations. Prototyping addresses Microsoft Azure Cloud as an
example. The prototyping outcome has been just published on GitHub as
the open-source (MIT licensed). The proposed solutions have been
harmonized with the more general concept called the Object-Oriented
Internet.

The main goal of this article is to provide proof that:

\begin{enumerate}
\def\labelenumi{\arabic{enumi}.}
\tightlist
\item
  reactive interoperability M2M communication based on the OPC UA
  standard can be implemented as a powerful standalone library without
  dependency on the Client/Server session-oriented archetype
\item
  Azure interoperability can be implemented as an external part
  employing out-of-band communication without dependency on the OPC UA
  implementation
\item
  the proposed generic architecture allows that the gateway
  functionality is composable at run-time - no programming required
\end{enumerate}

The remainder of this paper is structured as follows. Sect.
\texttt{Azure\ Main\ Technology\ \ Features} analyzes data presentation
user interface, available native communication services, and
data/metadata model offered by the Microsoft Azure. The discussion
covered by this section is the foundation for selecting services
utilized to expose process data and suitable protocol stack to support
interconnection. In Sect.
\texttt{Object-Oriented\ Internet\ Main\ Technology\ Features} the
discussion focuses on the generic architecture that is to be used as a
foundation for further decisions addressing the systematic design of the
interoperability of the cyber-physical systems and cloud-based
front-end. Sect.
\texttt{Azure\ to\ Sensors\ (A2S)\ connectivity\ deployment\ (field\ level\ connectivity)}
presents the proposed open and reusable software model. It promotes a
reactive interoperability pattern and a generic approach to establishing
interoperability-context. A reference implementation of this archetype
is described in Sect. \texttt{Gateway\ implementation}. The most
important findings and future work are summarized in Sect.
\texttt{Conclusions}.

\hypertarget{consider-to-add}{%
\subsection{Consider to add}\label{consider-to-add}}

\hypertarget{scope}{%
\subsubsection{Scope}\label{scope}}

What we must do to prove the goal have been achieved. Extent or range of
development, view, outlook, application, operation, effectiveness, etc.

\hypertarget{related-work}{%
\subsubsection{Related work}\label{related-work}}

Any information about available reusable deliverables related to this
work. \# Azure Main Technology Features

\hypertarget{services}{%
\subsection{Services}\label{services}}

Deployment of the hybrid solution providing interoperability of the
\textbf{machine-centric} cyber-physical systems designed atop of M2M
reactive communication and emerging cloud computing as a
\textbf{human-centric} front-end requires decisions addressing the
selection of the services supporting web user interface capable to
expose real-time process data. In this context, the service is any
autonomous (with own identity) software component or module that is
interfacing with selected cyber-physical systems for data collection,
analysis, and also remote control. Microsoft Azure is a cloud-based
product. It offers a vast variety of services. This virtual environment
handles an unlimited number of users and devices organized using a
solution concept. The solution aggregates users, devices, services, and
required additional resources scoping on a selected scenario. The
solution is a region that provides a scope to the identifiers (the names
of devices, users, process data entities, etc) inside it. Solutions are
used to organize deployment entities into logical groups and prevent
identity collisions.

The \texttt{IoT\ Central} service provides a process data visualization
user interface. To make this interface meaningful metadata called device
template is used to describe devices.

Following the assumption that interconnection between the cyber-physical
system and cloud services is designed based on the gateway concept, a
middleware must be considered aa a coupler. It must be interconnected
with the cyber-physical system using an in-bound protocol adhering to
communications requirements (i.e.~protocol profile, data encoding, time
relationships, etc.) governing communication of the parts making it up.
At the same time, it must support back-and-forth data transfer to the
cloud using out-of-bound native for the cloud services. The transfer
process requires data conversion from source to destination encoding.
The \texttt{IoT\ Hub} is a service hosted in the cloud that supports
\texttt{IoT\ Central} providing a robust messaging solution - it acts as
a central message hub for bi-directional communication. This
communication is transparent, i.e.~it is not data types aware allowing
any devices to exchange any kind of data. This service is responsible to
manage the devices' identity and it offers the following protocol
stacks: AMQP, MQTT, HTTPS.

Before process data can be exposed using a web user interface the data
source must be associated with an appropriate solution and validated to
make sure that the security rules are not violated. It is hard to assume
that the security rules governing the cyber-physical system may also
apply to the cloud-based services. In the gateway scenario, they can be
mapped on each other or entirely independent. The
\texttt{IoT\ Hub\ Device\ Provisioning\ Service} (DPS) is a helper
service for \texttt{IoT\ Hub} that enables devices registration,
authentication, and authorization of the requested operations including
but not limited to data transfer updating the user interface.

It is worth stressing that interaction of the offered by the Azure
services can be configured flexibly, and as a result, the presented
above selection of services must be recognized as an example only. The
`\texttt{IoT\ Central} can be also seamlessly integrated with other
services as needed. The following services could also be considered to
build cloud-based automation solution:

\begin{itemize}
\tightlist
\item
  \texttt{Industrial\ IoT} - discovering OPC UA enabled servers in a
  factory network and register them in Azure IoT Hub implemented using
  \texttt{IoT\ Edge}
\item
  \texttt{Digital\ Twins} - managing the graph of digital twins, which
  are to represent some real-world process or entity
\end{itemize}

\texttt{Industrial\ IoT} promotes OPC UA client/server archetype used to
achieve direct and interactive interoperability implemented using
\texttt{IoT\ Edge} services that allow extracting initial data
processing to local premises based on the edge concept.
\texttt{Digital\ Twins} is an emerging concept to use an observer to
replicate selected process state and behavior. The possibility to add
value as a result of using these services must be subject to further
research.

\hypertarget{data-interchange}{%
\subsection{Data Interchange}\label{data-interchange}}

System components interoperability means the necessity of the
information exchange between them. The main challenge of
interoperability implementation is that information is abstract -- it is
knowledge describing the process in concern state and behavior,
e.g.~temperature in a boiler, a car speed, an account balance, etc.
Obviously, abstraction cannot be processed by the cyber-physical
machines. It is also impossible to transfer abstraction from one place
to another over the network.

Fortunately, computer science offers a workaround to address that
impossibility - the information must be represented as a binary stream.
In consequence, we can usually use both ones as interchangeable terms
while talking about ICT systems. Unfortunately, these terms must be
precisely observed in the context of further discussion, because we must
be aware of the fact that the same information could have many different
but equivalent representations. In other words, the same information can
be represented by a vast variety of different binary patterns. For
example, numbers may be represented using 2's Complement and
Floating-Point binary representations.

It should be nothing new for us, as it is obvious that the same
information printed as a text in regional newspapers in English, German,
Polish, etc. does not resemble one another, but the text meaning should
always be the same. To understand a newspaper we must learn the
appropriate language. To understand the binary data we must have defined
a data type -- a description of how to create an appropriate bits
pattern (syntax) and rules needed to assign the information (semantics),
i.e.~make any correct bits stream meaningful. Concluding, to make two
systems interoperable, a semantic-context must be established. The type
plays the role of metadata, a set of data that describes other data.
Metadata term is frequently used if the semantic-context is defined
using a native language to select built-in types engaging a
general-purpose graphical user interface.

Using the data type definitions to describe information interchanged
between communicating parties allows:

\begin{itemize}
\tightlist
\item
  Development against a type definition of the user interface
\item
  Implementation of the functionality of the bits-streams conversion in
  advance
\end{itemize}

Having defined types in advance, a gateway may provide dedicated
conversions functionality, i.e.~replacing bits-stream used by the
cyber-physical system by equivalent once for the cloud-based services.
The Azure offers a vast variety of build-in types ready to be used in
common cases, but not necessarily there are equivalent counterparts in
use by the cyber-physical system. Additionally, the data conversion must
address the following issues:

\begin{itemize}
\tightlist
\item
  usually to covert data from source to destination representation, the
  middleware software native types must be used
\item
  if the out of the box set of types is not capable of fulfilling more
  demanding needs custom data types may be defined
\end{itemize}

Although the data conversion is a run-time gateway task the
implementation of the conversion algorithms must be recognized as an
engineering task, and therefore this topic is not considered for further
discussion.

In \texttt{IoT\ Central} a cyber-physical system is represented as a set
of devices. The characteristics and behaviors of each of a device kind
are described by the device template. This Device Template (DT)contains
also metadata describing the data (called telemetry) exchanged over the
wire with the cyber-physical system called Device Capability Model
(DCM). Additionally, the DT contains properties, customization, and
views definitions used by the service locally. As an option, DCM
expressed as a JSON can be imported into a Device Template.
\texttt{IoT\ Central} allows also to create or edit a DCM using the
dedicated web UI. A JSON file containing DCM can be derived from an
information model used as a foundation to establish the semantic-context
applied to achieve interoperability of the devices interconnected as the
cyber-physical system. DCM development against any external information
model is a design-time task and should be supported by dedicated
development tools. In any case, the data interchanged between the cloud
and the gateway must be compliant with the DCM, hence the middleware
software must be aware of conversions that must be applied to achieve
this interoperability.

\hypertarget{connectivity}{%
\subsection{Connectivity}\label{connectivity}}

From the cloud side, it is proposed to employ the \texttt{IoT\ Hub}
service to handle the network traffic targeting the cyber-physical
system. This service offers profiles of the AMQP, MQTT, HTTPS protocol
stacks. In any case, process data (telemetry) is transparently
transferred back-and-forth to the upper layer \texttt{IoT\ Central}
service. Hence, the payload formatting is determined by the DCM
associated with the \texttt{IoT\ Central} solution. All the mentioned
protocols are standard ones. Consequently, it is possible to apply any
available implementation compliant with an appropriate specification to
achieve connectivity. In this case, all parameters required to establish
connectivity and security-context is up to the external software
responsibility. Alternatively, the API offered by the dedicated
libraries may be used. Using this API the configuration process can be
reduced significantly. Using these libraries, the selection of the
communication protocol has an indirect impact on the interoperability
features, including performance. The connectivity with
\texttt{IoT\ Hub}, for example, can be obtained using:

\begin{itemize}
\tightlist
\item
  \texttt{Microsoft.Azure.Devices} - Service SDK for Azure IoT Devices
\item
  \texttt{Microsoft.Azure.Devices.Client}- Device SDK for Azure IoT Hub
\item
  \texttt{Microsoft.Azure.Devices.Provisioning.Client} - Provisioning
  Device Client SDK for Azure IoT Devices
\item
  \texttt{Microsoft.Azure.Devices.Provisioning.Transport.Amqp} -
  Provisioning Device Client AMQP Transport for Azure IoT Devices
\item
  \texttt{Microsoft.Azure.Devices.Provisioning.Transport.Http} -
  Provisioning Device Client Http Transport for Azure IoT Devices
\item
  \texttt{Microsoft.Azure.Devices.Provisioning.Transport.Mqtt} -
  Provisioning Device Client MQTT Transport for Azure IoT Devices
\item
  \texttt{Microsoft.Azure.Devices.Shared} - Common code for Azure IoT
  Device and Service SDKs
\end{itemize}

\hypertarget{ooi-main-technology-features}{%
\section{OOI Main Technology
Features}\label{ooi-main-technology-features}}

\begin{itemize}
\tightlist
\item
  Machinie To Machine communication based on the semantic data
\item
  OOI PubSub Implementation Architecture
\item
  Simple, complex and structural data processing
\end{itemize}

Any information about available reusable deliverables related to this
work.

\hypertarget{iot}{%
\subsection{IoT}\label{iot}}

Internet of Things is all about

\begin{itemize}
\tightlist
\item
  Mobile data fetching -- how to gather the data from mobile devices
  (things)
\item
  Mobile data subscription -- how to transfer the data over the Internet
  to a place where it could be subscribed.
\item
  Mobile data consumption -- data processing, computation
\end{itemize}

Mobile data fetching is related to variety of last mile communication
technologies, for example RFID, WI-FI, VHF, etc. Subscription could be
supported using messaging systems, e.g.~AMQP. A good candidate to
leverage consumption is OPC Unified Architecture.

To deploy this scenario

\begin{itemize}
\tightlist
\item
  the mobile data must be sent over the Internet using messages;
\item
  the payload of these messages is consumed by an OPC Server responsible
  to expose it in the Address Space;
\item
  the application as an OPC UA client consumes the exposed data and
  generates revenue for the end-user.The phrase ``Internet of Things''
  started its life as the title of a presentation made in 1999 and aimed
  at explaining a new idea of radio frequency identification (RFID) in
  the context of the supply chain performance. It is clear that it
  doesn't mean that someone has any right to control how others use the
  phrase, but my point is that a precise term definition is important
  for working together on: common rules, architecture, solutions,
  requirements, capabilities, limitations, etc. In practice having a
  common definition it is possible to check a selected technology,
  solution or product capabilities against requirements of the
  application entitled to use this term.
\end{itemize}

My proposal of the Internet of Things definition is as follows:

Internet of Things is all about:

\begin{itemize}
\tightlist
\item
  mobile data fetching -- how to gather the data from mobile devices
  (things);
\item
  mobile data subscription -- how to transfer the data over the Internet
  to a place where it could be processed;
\item
  mobile data processing -- how to integrate the data into a selected
  application to improve process behavioral performance.
\end{itemize}

Data fetching is related to a variety of last mile communication
technologies, for example RFID, WI-FI, VHF, Bluetooth, etc. Subscription
could be supported using messaging systems, e.g.~AMQP, MQTT, etc. A good
candidate for leveraging data consumption is for example OPC Unified
Architecture.

\hypertarget{azure-to-sensors-a2s-connectivity-deployment-field-level-connectivity}{%
\section{Azure to Sensors (A2S) connectivity deployment (field level
connectivity)}\label{azure-to-sensors-a2s-connectivity-deployment-field-level-connectivity}}

\begin{quote}
The title must be revised
\end{quote}

\begin{itemize}
\tightlist
\item
  \textbf{Architecture} - Domain model presenting relationship between
  the: Azure, PubSub Gateway, Device, Design and development tools
\item
  \textbf{Connectivity} - Describe reactive nature of the Azure
  monitoring process data (telemetry) services.
\item
  \textbf{Deployment phases}

  \begin{itemize}
  \tightlist
  \item
    Design
  \item
    Gateway and devices registration
  \item
    Authentication
  \item
    Device/Service association
  \item
    Device/Application association
  \item
    Establishing session

    \begin{itemize}
    \tightlist
    \item
      Device/Device Template (Device Capability Model) association -
      establishing a semantic-context
    \item
      Security management - establishing security-context
    \end{itemize}
  \item
    Interconnection - exchange of data
  \item
    Maintenance
  \end{itemize}
\end{itemize}

We have selected
\href{https://docs.microsoft.com/azure/iot-central/core/}{IoT Central}
because:

\begin{itemize}
\tightlist
\item
  provides process data visualization user interface
\item
  allows to describe devices using metadata containing telemetry data
  types
\end{itemize}

\hypertarget{gateway-implementation}{%
\section{Gateway implementation}\label{gateway-implementation}}

\begin{itemize}
\tightlist
\item
  Architecture
\item
  Protocol selection and mapping
\item
  Configuration
\item
  Testing
\end{itemize}

\hypertarget{conclusion}{%
\section{Conclusion}\label{conclusion}}

The OPC UA PubSub to Azure gateway (\texttt{AzureGateway})
implementation has been just published on GitHub as the open-source (MIT
licensed) as a part of the more general concept of the Object-Oriented
Internet reactive networking. It is proof of the concept that

\begin{enumerate}
\def\labelenumi{\arabic{enumi}.}
\tightlist
\item
  OPC UA PubSub can be implemented as a powerful standalone package - no
  C/S dependency
\item
  Azure interoperability can be implemented as an out-of-band
  communication (MQTT, AMQP, HTTP) - no PubSub dependency
\item
  Process data functionality can be composable at run-time - no
  programming required
\end{enumerate}

\end{document}
