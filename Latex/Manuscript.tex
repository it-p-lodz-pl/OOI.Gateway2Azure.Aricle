\documentclass{jacsart}
\usepackage{hyperref}
\usepackage{cite}
\usepackage{stfloats}
\usepackage{url}
\usepackage{authblk}
\usepackage[utf8]{inputenc}
\usepackage[T1]{fontenc}
\usepackage{graphicx}
\usepackage{amsthm}
\usepackage{txfonts}

\newtheorem{definition}{Definition}
\newtheorem{theorem}[definition]{Theorem}
\newtheorem{corollary}[definition]{Corollary}
\newtheorem{proposition}[definition]{Proposition}
\newtheorem{example}[definition]{Example}


\providecommand{\tightlist} { \setlength{\itemsep}{0pt}\setlength{\parskip}{0pt}}
\providecommand{\keywords}[1]{  \small	  \textbf{\textit{Keywords---}} #1 }

\title{Object-Oriented Internet - Azure interoperability}
% \title{Object-Oriented Internet - reactive visualization of asynchronous data using AZURE}
\headtitle{OOI - Azure interoperability}

%\author[1]{Mariusz Postół}
%\author[2]{Piotr Szymczak}
%\author[3]{Clemens Vasters}
%\affil[1, 2]{Institute of Information Technology, Lodz University of Technology, Łódź, Poland}
% \email{mailto:mariusz.postol@p.lodz.pl} 
%\affil[3]{Microsort}
%\tocauthor{Mariusz Postół}
%\institute{ Institute of Information Technology, Lodz University of Technology, Łódź, Poland \email{mailto:mariusz.postol@p.lodz.pl} }

\author{Author One\inst{1}, Author Two\inst{1,2}}
\headauthor{A. One, A. Two}
\affiliation{%
  \inst{1}Name of the Unit represented (e.g. Your University)\\
  Faculty/Department/Office Name\\
  Postal Addres with the zip-code\\
  yourid@your.mail.server
  \andinst
  \inst{2}Name of the Unit represented (e.g. Your University)\\
  Faculty/Department/Office Name\\
  Postal Addres with the zip-code\\
  yourid@your.mail.server}

\keywords{Azure, Cloud Computing, Object-Oriented Internet, Industrial communication, Industry 4.0, Internet of Things, Machine to Machine Communication, OPC Unified Architecture}

\begin{document} 

\maketitle


\begin{abstract}
Each paper should be followed by a compact abstract which points to the main
scopes and the results obtained by the paper. The abstract should be written
with the Times New Roman font, 10 pt, justified, and with the 1cm margin both
left and right side with respect  to the margin of the paper. The abstract should not contain any formulas or references, and should not exceed 200 words. 
\end{abstract}

\maketitle

\hypertarget{introduction}{%
  \section{Introduction}\label{introduction}}

\begin{itemize}
  \tightlist
  \item
        \textbf{Subject} - A basic matter of thought, discussion,
        investigation, development, etc. Describe the problem and the
        motivation for undertaking the effort to solve the problem.
  \item
        \textbf{Goal} What we are going to achieve - the result or achievement
        toward which effort is directed.
  \item
        \textbf{Scope} - What we must do to prove the goal have been achieved.
        Extent or range of development, view, outlook, application, operation,
        effectiveness, etc.
  \item
        \textbf{Related work} - Any information about available reusable
        deliverables related to this work.
\end{itemize}

\hypertarget{azure-main-technology-features}{%
  \section{Azure Main Technology
    Features}\label{azure-main-technology-features}}

\begin{itemize}
  \tightlist
  \item
        \textbf{Selection of the service}
  \item
        \textbf{Metadata} - must be discussed in context of the design/run
        time stages.

        \begin{itemize}
          \tightlist
          \item
                \textbf{Device Template (DT)}
          \item
                \textbf{Device Capability Model}
          \item
                \textbf{Interface}
          \item
                \textbf{Digital Twin Definition Language (DTDL)}
        \end{itemize}
  \item
        \textbf{Simple, complex and structural data processing}
  \item
        \textbf{Connectivity}
  \item
        \textbf{How to implement} All about available libraries and tools
\end{itemize}

\hypertarget{ooi-main-technology-features}{%
  \section{OOI Main Technology
    Features}\label{ooi-main-technology-features}}

\begin{itemize}
  \tightlist
  \item
        Machinie To Machine communication based on the semantic data
  \item
        OOI PubSub Implementation Architecture
  \item
        Simple, complex and structural data processing
\end{itemize}

\hypertarget{azure-to-sensors-a2s-connectivity-deployment-field-level-connectivity}{%
  \section{Azure to Sensors (A2S) connectivity deployment (field level
    connectivity)}\label{azure-to-sensors-a2s-connectivity-deployment-field-level-connectivity}}

\begin{quote}
  The title must be revised
\end{quote}

\begin{itemize}
  \tightlist
  \item
        \textbf{Architecture} - Domain model presenting relationship between
        the: Azure, PubSub Gateway, Device, Design and development tools
  \item
        \textbf{Connectivity} - Describe reactive nature of the Azure
        monitoring process data (telemetry) services.
  \item
        \textbf{Deployment phases}

        \begin{itemize}
          \tightlist
          \item
                Design
          \item
                Gateway and devices registration
          \item
                Authentication
          \item
                Device/Service association
          \item
                Device/Application association
          \item
                Establishing session

                \begin{itemize}
                  \tightlist
                  \item
                        Device/Device Template (Device Capability Model) association -
                        establishing a semantic-context
                  \item
                        Security management - establishing security-context
                \end{itemize}
          \item
                Interconnection - exchange of data
          \item
                Maintenance
        \end{itemize}
\end{itemize}

We have selected
\href{https://docs.microsoft.com/azure/iot-central/core/}{IoT Central}
because:

\begin{itemize}
  \tightlist
  \item
        provides process data visualization user interface
  \item
        allows to describe devices using metadata containing telemetry data
        types
\end{itemize}

\hypertarget{gateway-implementation}{%
  \section{Gateway implementation}\label{gateway-implementation}}

\begin{itemize}
  \tightlist
  \item
        Architecture
  \item
        Protocol selection and mapping
  \item
        Configuration
  \item
        Testing
\end{itemize}

\hypertarget{conclusion}{%
  \section{Conclusion}\label{conclusion}}

The OPC UA PubSub to Azure gateway (\texttt{AzureGateway})
implementation has been just published on GitHub as the open-source (MIT
licensed) as a part of the more general concept of the Object-Oriented
Internet reactive networking. It is proof of the concept that

\begin{enumerate}
  \def\labelenumi{\arabic{enumi}.}
  \tightlist
  \item
        OPC UA PubSub can be implemented as a powerful standalone package - no
        C/S dependency
  \item
        Azure interoperability can be implemented as an out-of-band
        communication (MQTT, AMQP, HTTP) - no PubSub dependency
  \item
        Process data functionality can be composable at run-time - no
        programming required
\end{enumerate}

\bibliography{Manuscript}
\bibliographystyle{aiaa-jacs.bst}

\end{document}
