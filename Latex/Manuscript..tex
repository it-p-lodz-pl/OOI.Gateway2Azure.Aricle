\documentclass{article}
\usepackage{hyperref}
\usepackage{cite}
\usepackage{graphicx}
\usepackage{stfloats}
\usepackage{url}
\usepackage{authblk}

\providecommand{\tightlist} { \setlength{\itemsep}{0pt}\setlength{\parskip}{0pt}}

\title{Object-Oriented Internet - Azure interoperability} 
\author[1]{Mariusz Postół}
\author[2]{Piotr Szymczak}
\author[3]{Clemens Vasters}
\affil[1, 2]{Institute of Information Technology, Lodz University of Technology, Łódź, Poland}
% \email{mailto:mariusz.postol@p.lodz.pl} 
\affil[3]{Microsort}
%\tocauthor{Mariusz Postół}
%\institute{ Institute of Information Technology, Lodz University of Technology, Łódź, Poland \email{mailto:mariusz.postol@p.lodz.pl} }

\begin{document} 

\maketitle


\begin{abstract}

  Information and Communication Technology has provided society with a~vast variety of distributed applications. By design, the deployment of this kind of application has to focus primarily on communication. This article addresses research results on the systematic approach to the design of the meaningful Machine to Machine (M2M) communication targeting distributed mobile applications in the context of new emerging disciplines, i.e. Industry 4.0 and Internet of Things. This paper contributes to the design of a new architecture of mobile IoT solutions designed atop of the M2M communication and composed as multi-vendor cyber-physicals systems. The described reusable library supporting this architecture designed using the reactive interoperability archetype proves that the concept enables a systematic approach to the development and deployment of software applications against mobile IoT solutions based on international standards. Dependency injection and adaptive programming engineering techniques have been engaged to develop a full-featured reference application program and make the proposed solution scalable and robust against deployment environment continuous modifications. The article presents an executive summary of the proof of concept and describes selected conceptual and experimental results achieved as an outcome of the open-source project Object-Oriented Internet targeting multi-vendor plug-and-produce interoperability scenario.

%\keywords{Industrial communication \and Industry 4.0 \and Internet of Things \and Machine to Machine Communication \and OPC Unified Architecture}

\end{abstract}

\maketitle

\hypertarget{document-header}{%
  \section{Document Header}\label{document-header}}

\hypertarget{title-proposals}{%
  \subsection{Title proposals}\label{title-proposals}}

\begin{itemize}
  \tightlist
  \item
        Object-Oriented Internet - reactive visualization of asynchronous data
        using AZURE
  \item
        Object-Oriented Internet - Azure interoperability
\end{itemize}

\hypertarget{abstract}{%
  \subsection{Abstract}\label{abstract}}

\hypertarget{key-words}{%
  \subsection{Key words}\label{key-words}}

Azure, Cloud Computing, Object-Oriented Internet, OPC Unified
Architecture, Machine to Machine Communication, Internet of Things, OPC
UA, RxNetworking

\hypertarget{introduction}{%
  \section{Introduction}\label{introduction}}

\begin{itemize}
  \tightlist
  \item
        \textbf{Subject} - A basic matter of thought, discussion,
        investigation, development, etc. Describe the problem and the
        motivation for undertaking the effort to solve the problem.
  \item
        \textbf{Goal} What we are going to achieve - the result or achievement
        toward which effort is directed.
  \item
        \textbf{Scope} - What we must do to prove the goal have been achieved.
        Extent or range of development, view, outlook, application, operation,
        effectiveness, etc.
  \item
        \textbf{Related work} - Any information about available reusable
        deliverables related to this work.
\end{itemize}

\hypertarget{azure-main-technology-features}{%
  \section{Azure Main Technology
    Features}\label{azure-main-technology-features}}

\begin{itemize}
  \tightlist
  \item
        \textbf{Selection of the service}
  \item
        \textbf{Metadata} - must be discussed in context of the design/run
        time stages.

        \begin{itemize}
          \tightlist
          \item
                \textbf{Device Template (DT)}
          \item
                \textbf{Device Capability Model}
          \item
                \textbf{Interface}
          \item
                \textbf{Digital Twin Definition Language (DTDL)}
        \end{itemize}
  \item
        \textbf{Simple, complex and structural data processing}
  \item
        \textbf{Connectivity}
  \item
        \textbf{How to implement} All about available libraries and tools
\end{itemize}

\hypertarget{ooi-main-technology-features}{%
  \section{OOI Main Technology
    Features}\label{ooi-main-technology-features}}

\begin{itemize}
  \tightlist
  \item
        Machinie To Machine communication based on the semantic data
  \item
        OOI PubSub Implementation Architecture
  \item
        Simple, complex and structural data processing
\end{itemize}

\hypertarget{azure-to-sensors-a2s-connectivity-deployment-field-level-connectivity}{%
  \section{Azure to Sensors (A2S) connectivity deployment (field level
    connectivity)}\label{azure-to-sensors-a2s-connectivity-deployment-field-level-connectivity}}

\begin{quote}
  The title must be revised
\end{quote}

\begin{itemize}
  \tightlist
  \item
        \textbf{Architecture} - Domain model presenting relationship between
        the: Azure, PubSub Gateway, Device, Design and development tools
  \item
        \textbf{Connectivity} - Describe reactive nature of the Azure
        monitoring process data (telemetry) services.
  \item
        \textbf{Deployment phases}

        \begin{itemize}
          \tightlist
          \item
                Design
          \item
                Gateway and devices registration
          \item
                Authentication
          \item
                Device/Service association
          \item
                Device/Application association
          \item
                Establishing session

                \begin{itemize}
                  \tightlist
                  \item
                        Device/Device Template (Device Capability Model) association -
                        establishing a semantic-context
                  \item
                        Security management - establishing security-context
                \end{itemize}
          \item
                Interconnection - exchange of data
          \item
                Maintenance
        \end{itemize}
\end{itemize}

We have selected
\href{https://docs.microsoft.com/azure/iot-central/core/}{IoT Central}
because:

\begin{itemize}
  \tightlist
  \item
        provides process data visualization user interface
  \item
        allows to describe devices using metadata containing telemetry data
        types
\end{itemize}

\hypertarget{gateway-implementation}{%
  \section{Gateway implementation}\label{gateway-implementation}}

\begin{itemize}
  \tightlist
  \item
        Architecture
  \item
        Protocol selection and mapping
  \item
        Configuration
  \item
        Testing
\end{itemize}

\hypertarget{conclusion}{%
  \section{Conclusion}\label{conclusion}}

The OPC UA PubSub to Azure gateway (\texttt{AzureGateway})
implementation has been just published on GitHub as the open-source (MIT
licensed) as a part of the more general concept of the Object-Oriented
Internet reactive networking. It is proof of the concept that

\begin{enumerate}
  \def\labelenumi{\arabic{enumi}.}
  \tightlist
  \item
        OPC UA PubSub can be implemented as a powerful standalone package - no
        C/S dependency
  \item
        Azure interoperability can be implemented as an out-of-band
        communication (MQTT, AMQP, HTTP) - no PubSub dependency
  \item
        Process data functionality can be composable at run-time - no
        programming required
\end{enumerate}

\bibliography{Manuscript}
\bibliographystyle{splncs04}

\end{document}
